\ifx\allfiles\undefined
\documentclass[12pt, a4paper, oneside, UTF8]{ctexbook}
\def\path{../config}
\input{\path/package.tex}
\input{\path/theorem2_zh.tex}
\input{\path/custom.tex}

\def\myIndex{0}  % 封面
% \input{\path/cover_package_\myIndex.tex}

\def\myTitle{数学物理方程}
\def\myAuthor{高峰}
\def\myDateCover{2020 年 12 月 16 日\ --}
\def\myDateForeword{\today}
\def\myForeword{前言}
\def\myForewordText{
    
}
\def\mySubheading{}


\begin{document}
% \input{../config/cover}
\else
\fi

\chapter{分离变数法}

\section{齐次方程的分离变数法}

一维齐次波动方程的混合问题:
\begin{equation}
    \left\{ 
    \begin{lgathered}
        u_{tt} - a^2u_{xx} = 0, \quad 0 < x < l,\;t > 0,  \\
        u|_{t = 0} = \varphi(x),\;u_{t}|_{t = 0} = \psi(x),\quad 0 \leqslant x \leqslant l,  \\
        u|_{x = 0} = 0,\;u|_{x = l} = 0,\quad t \geqslant 0.
    \end{lgathered}   
    \right.
\end{equation}
(其中 $\varphi(0)=\varphi(l)=0$,这个条件也被称为相容性条件)

该定解问题的解为:
\begin{equation}
    u(x,t) = \sum^\infty_{k = 1}(A_k\cos \frac{k\/\mpi a}{l} t 
        + B_k\sin\frac{k\mpi a}{l} t\,)\sin \frac{k\mpi}{l}x,
\end{equation}
\begin{equation}
    \mbox{其中,\ }
    \left\{
    \begin{lgathered}    
        A_k = \frac{2}{l}\int^l_0 \varphi(\xi)\sin \frac{k\mpi\xi}{l}\dif \xi,  \\
        B_k = \frac{2}{k\mpi a}\int^l_0 \psi (\xi)\sin \frac{k\mpi\xi}{l}\dif \xi.
    \end{lgathered}
    \right.
\end{equation}
\colorstar 解是很多模的叠加,每一个模可以分成单元函数的积 

\colorstar 抛物型方程(输运过程)无法从当前反推早先时刻的状态

\colorstar 定解问题的解在实际问题中常常前若干项较为重要,后面的项因太小可略去。



\ifx\allfiles\undefined
\end{document}
\fi