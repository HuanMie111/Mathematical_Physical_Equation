\ifx\allfiles\undefined
\documentclass[12pt, a4paper, oneside, UTF8]{ctexbook}
\def\path{../config}
\input{\path/package.tex}
\input{\path/theorem2_zh.tex}
\input{\path/custom.tex}

\def\myIndex{0}  % 封面
% \input{\path/cover_package_\myIndex.tex}

\def\myTitle{数学物理方程}
\def\myAuthor{高峰}
\def\myDateCover{2020 年 12 月 16 日\ --}
\def\myDateForeword{\today}
\def\myForeword{前言}
\def\myForewordText{
    
}
\def\mySubheading{}


\begin{document}
% \input{../config/cover}
\else
\fi

\chapter{数学物理定解问题}

\section{基本概念}

方程:含有未知量的等式。(\hyperref[fig:方程分类]{Fig.~\ref{fig:方程分类}})

\begin{figure}
    \centering
    \includegraphics{方程分类.pdf} 
    \caption{\label{fig:方程分类} 方程分类}
\end{figure}
\begin{defn}{偏微分函数}{}
    关于多元函数 $ u(x,y,\cdots) $ 及其某些偏导数的关系式:
    \[ 
        F(x, y, \cdots, u, u_x, u_y, \cdots, u_{xx}, u_{xy}, u_{yy}, \cdots, 
        u_{xxx}, u_{xxy},\cdots) = 0, 
    \]
    其中 $F$ 是 $x$, $y$, $\cdots$, $u$ 及 $u$ 的有限多个偏导数的已知函数。
\end{defn}
\begin{rmk}{}
    偏微分方程(1)未知函数是多元函数,且未知数个数有限;
    (2)包含未知函数的某些偏导数的方程。
\end{rmk}

偏微分方程的研究集中在少数特殊类型的偏微分方程,一般性较少,个性多。

偏微分方程的解  \\
\textcolor{red}{偏微分方程的通解是含有自变量的任意函数}

偏微分方程的阶
% TODO: 使用红色下划线
\noindent 线性偏微分方程(\hyperref[fig:偏微分方程分类]{Fig.~\ref{fig:偏微分方程分类}})
:方程中实际所含未知数的各阶偏导数是线性的。  \\
自由项:在\textcolor{red}{线性偏微分方程}中,不含 $u$ 及它的偏导数的项。  \\
线性方程中,自由项为 0 的是齐次偏微分方程,自由项不为 0 的是非齐次偏微分方程。  \\
线性方程的叠加原理  \\
非线性偏微分方程:不是线性的统称非线性。  \\
拟线性偏微分方程:在非线性偏微分方程中,关于未知函数的所有最高阶偏导数都是线性的。  \\
主部:在拟线性偏微分方程中,由最高阶偏导数组成的部分。  \\
半线性:主部的系数都是常数或是自变量的已知函数。  \\
完全非线性偏微分方程:既不是线性,也不是拟线性的偏微分方程。

\begin{figure}
    \centering
    \includegraphics{偏微分方程分类.pdf}
    \caption{\label{fig:偏微分方程分类} 偏微分方程分类}
\end{figure}

\section{二阶偏微分方程}

二阶 PDE 的分类 (\hyperref[fig:二阶PDE]{Fig.~\ref{fig:二阶PDE}})  \\
二阶线性 PDE 的判别式 $ \Delta = a_{12}^2-a_{11}a_{22} 
    \begin{cases} > 0 & \mbox{双曲型}  \\  
        = 0 & \mbox{抛物型}  \\
        < 0 & \mbox{椭圆型}
    \end{cases} $  \\
二阶线性 PDE 的特征方程:
\[ a_{11}(\frac{\dif y}{\dif x})^2 - 2a_{12}\frac{\dif y}{\dif x} + a_{22} = 0\]
\colorstar 二阶 PDE 化简的一个方法  \\
原方程为 $a_{11}u_{xx} + 2a_{12}u_{xy} + a_{22}u_{yy}
    + b_1u_x + b_2u_y + cu = f$,则:
\[ \begin{bmatrix}
        \overline{a_{11}} & \overline{a_{12}}  \\
        \overline{a_{21}} & \overline{a_{22}}
    \end{bmatrix} = \bit{Q} 
    \begin{bmatrix}
        a_{11} & a_{12}  \\
        a_{21} & a_{22}
    \end{bmatrix} \bit{Q}^T,\ \mbox{其中}\ \bit{Q} = 
    \begin{bmatrix}
        \xi_{x} & \xi_{y}  \\
        \eta_{x} & \eta_{y}
    \end{bmatrix},\]
\[ \overline{b_{1}} = (L - c)\xi,\quad \overline{b_{2}} = (L - c)\eta,\quad
    \overline{c} = c,\quad \overline{f} = f,\]
其中 $ L = a_{11}\frac{\partial ^{2}}{\partial ^{2} x^{2}}
    + 2a_{12}\frac{\partial ^{2}}{{\partial {x}}{\partial {y}}}
    + a_{22}\frac{\partial ^{2}}{\partial ^{2} y^{2}} 
    + b_{1}\frac{\partial}{\partial x}
    + b_{2}\frac{\partial}{\partial y} + c,$
$\overline{b_{1}}$ 等为新方程的对应参数。

\begin{figure}
    \centering
    \includegraphics{二阶PDE.pdf}
    \caption{\label{fig:二阶PDE} 二阶PDE}
\end{figure}

\noindent \colorstar 拉普拉斯 Laplace 算符:

一维 $\Delta_1 = \nabla\cdot\nabla
    = \frac{\partial ^{2}}{\partial ^{2} x^{2}}$
    
二维 $\Delta_2 = \nabla\cdot\nabla
    = \frac{\partial ^{2}}{\partial ^{2} x^{2}}
    + \frac{\partial ^{2}}{\partial ^{2} y^{2}}$
    
三维 $\Delta_3 = \nabla\cdot\nabla = \nabla^2 
    = \frac{\partial ^{2}}{\partial ^{2} x^{2}}
    + \frac{\partial ^{2}}{\partial ^{2} y^{2}}
    + \frac{\partial ^{2}}{\partial ^{2} z^{2}}$
    
四维 $\Delta_4 = \square\cdot\square
    = \frac{\partial ^{2}}{\partial ^{2} x^{2}}
    + \frac{\partial ^{2}}{\partial ^{2} y^{2}}
    + \frac{\partial ^{2}}{\partial ^{2} z^{2}}
    - \frac{1}{c^2}\frac{\partial ^{2}}{\partial ^{2} t^{2}}$。

(\textcolor{blue}{四维的 Laplace 算符就是达朗贝尔算符})

\section{定解问题}

\begin{equation*}
    \mbox{泛定方程} + \mbox{定解条件}
        \left\{ 
            \begin{lgathered} \mbox{初始条件} \\ \mbox{边界条件} \end{lgathered}   
        \right.
    = \mbox{定解问题}
        \left\{
            \begin{lgathered} 
                \mbox{初值问题(也叫做 Cauthy 问题)} \\ \mbox{边值问题} \\ \mbox{混合问题} 
            \end{lgathered} 
        \right.
\end{equation*}
泛定方程(\textcolor{red}{描绘普遍规律的方程});
Cauthy 问题(\textcolor{blue}{只有初始条件,没有(无需)边界条件});
\textcolor{blue}{条件可以为分段函数}。 \\
\colorstar $\mbox{对时间求导的阶数} = \mbox{所需的初始条件个数} $

守恒律:对任意的 $Q$,$Q$ 内的该物质的变化率 $=$ 净流入 $+$ $Q$内的总生成率

\noindent 例如:$u_t = -\;\Div{\bit{\Phi}} + f(x, t, u)$。  \\  % TODO div 前空格
守恒律是对各种物理过程普适的方程(共性)。
\[ \mbox{偏微分方程} = \mbox{守恒律} + \mbox{本构关系(特性定律)} \]

适定性:(1)存在性;(2)唯一性;(3)稳定性。

\noindent 三类典型的边界条件(以热传导方程为例)  \\
(1) 第一边值问题或狄利克雷问题  \\
边界条件:
 \[ u|_\varGamma = \varphi(x,y,z,t),\quad \varphi\ \mbox{为定义在}\ (x,y,z)\in \varGamma,\;
0\leqslant t \leqslant T\ \mbox{上的已知函数。} \]
\textcolor{blue}{$\varphi=0$时称为齐次狄利克雷条件,对诺伊曼条件、罗宾条件同样。}  \\
(2) 第二边值问题或诺伊曼边值问题
\[ (-\kappa \frac{\partial{u}}{\partial{\nu}})|_\varGamma = q(x,y,z,t),\quad 0\leqslant t \leqslant T,\]
\[ \frac{\partial{u}}{\partial{\nu}}|_\varGamma = \varphi(x,y,z,t),\] 
其中 $\varphi=-\frac{q}{\kappa}$ 是定义在 $\varGamma$,$0 \leqslant t \leqslant T$ 上的已知函数。  \\
\colorstar 绝热边界 $ \frac{\partial{u}}{\partial{x}}|_{x=x_0} = 0 $  \\
(3) 第三边值问题或罗宾边值问题
\[ (\frac{\partial{u}}{\partial{\nu}}+\sigma u)|_\varGamma = \varphi(x,y,z,t). \]

\noindent 其他边界条件  \\
(1) 衔接条件:例如 \textcolor{blue}{$E_{1t}=E_{2t}$}  \\
(2) 周期性条件:对时间或空间或其他变量  \\
(3) 自然 BC:\textcolor{blue}{无穷远}和\textcolor{blue}{中心}  \\
(大多数物理过程是有界和有限的)

零边界条件(齐次 BC) Vanishing boundary condition VBC;

非零边界条件(非齐次 BC) non-Vanishing boundary condition NVBC。

方法:行波法 \ding{73};积分变换法\ding{73};变分法;
分离变量法 \ding{73};格林函数法 \ding{73};数值法

\section{达朗贝尔公式}

一维无界的自由波动
\begin{equation}
    \left\{ 
    \begin{lgathered}
        u_{tt} - a^2u_{xx} = f(x,t)  \\
        u|_{t = 0} = \varphi(x),\;u_{t}|_{t = 0} = \psi(x),\quad x \in \Rs.
    \end{lgathered}   
    \right.
\end{equation}
达朗贝尔公式:
\begin{equation}
    u(x,t) = \frac{1}{2}[\varphi(x - at) + \varphi(x + at)] 
            + \frac{1}{2a}\int_{x - at}^{x + at} \psi(x) \dif x.
    \label{eq:达朗贝尔公式}
\end{equation}
\textcolor{red}{行波法}  \\
\colorstar 根据边界条件来确定延拓函数形式


\ifx\allfiles\undefined
\end{document}
\fi